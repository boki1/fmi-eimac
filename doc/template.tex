\definecolor{salmon}{RGB}{255, 128, 128}
\definecolor{redish}{HTML}{A20025}
\definecolor{darkgrey}{RGB}{50, 50, 50}

\setbeamercolor{frametitle}{fg=darkgrey}
\setbeamercolor{title separator}{fg=salmon}
\setbeamercolor{footline}{fg=darkgrey}
\setbeamercolor{progress bar}{fg=salmon}
\setbeamercolor{alerted text}{fg=salmon}
\setbeamercolor{block title example}{fg=salmon}
\setbeamercolor{section in foot}{fg=white, bg=salmon}

\makeatother
\setbeamertemplate{footline}
{
  \leavevmode%
  \hbox{%
  \begin{beamercolorbox}[wd=.333333\paperwidth,ht=2.25ex,dp=1ex,center]{section in foot}%
    \usebeamerfont{title in foot}К. Стоименов
  \end{beamercolorbox}%
  \begin{beamercolorbox}[wd=.333333\paperwidth,ht=2.25ex,dp=1ex,center]{section in foot}%
      \usebeamerfont{title in foot}\insertshorttitle
  \end{beamercolorbox}%
  \begin{beamercolorbox}[wd=.333333\paperwidth,ht=2.25ex,dp=1ex,right]{section in foot}%
    \insertframenumber{} / \inserttotalframenumber\hspace*{2ex}
  \end{beamercolorbox}}%
  \vskip0pt%
}
\makeatletter
\setbeamertemplate{navigation symbols}{}

\setbeamertemplate{frametitle}
{
\vspace*{16pt}
\Large{\insertframetitle}
% \insertprogressbar
}

\setbeamertemplate{section in toc}{{\color{salmon}\inserttocsectionnumber.}~\inserttocsection}

\setbeamertemplate{caption}{\raggedright\insertcaption\par}


\renewcommand*{\bibfont}{\footnotesize}

\newcommand{\iu}{{i\mkern1mu}}
\newcommand{\smiley}{\faSmileO}
\newcommand{\rootsofunity}[1] {
\begin{tikzpicture}
  \coordinate (center) at (0, 0);
  \def\radius{2cm}
  \def\n{#1}
  \pgfmathsetmacro\angle{360/\n}
  \pgfmathsetmacro\startangle{90}

  \foreach \i in {0,...,\the\numexpr\n-1\relax} {
    \coordinate (point\i) at ({\startangle+\angle*\i}:\radius);
    \coordinate (label\i) at ({\startangle+\angle*\i}:\radius+0.25cm);
    \node[anchor={\startangle+\angle*\i-90}] at (label\i){$\omega_{\n}^{\the\numexpr\n-\i\relax}$};
    \fill[salmon] (point\i) circle (2pt);
  }

  \draw[salmon, thick] (point0) \foreach \i in {1,...,\the\numexpr\n-1\relax} { -- (point\i) } -- cycle;
  \draw[black, thick] (center) circle (\radius);
  \draw[->] (-2.5,0) -- (2.5,0) node[right] {$\text{x}$};
  \draw[->] (0,-2.5) -- (0,2.5) node[above] {$\text{y}$};
\end{tikzpicture}
}

\newcommand{\egmarker}{
	\begin{tikzpicture}[remember picture,overlay]
		\node[anchor=north west, xshift=20pt, yshift=0pt] at (current page.north west) 
		{\tiny \textcolor{gray}{\raisebox{0.65em}{\rotatebox{270}{$\blacktriangle$}}~Пример}};
	\end{tikzpicture}
	
	\begin{tikzpicture}[remember picture,overlay]
		\draw[rounded corners=0.2em, line width=0.5pt, gray]
		([xshift=10pt,yshift=-10pt]current page.north west) 
		rectangle 
		([xshift=-10pt,yshift=20pt]current page.south east);
	\end{tikzpicture}
	\vspace*{-1.5em}
}
\newcommand{\demomarker}{
	\begin{tikzpicture}[remember picture,overlay]
		\node[anchor=north west, xshift=20pt, yshift=0pt] at (current page.north west) 
		{\tiny \textcolor{darkgray}{\raisebox{0.65em}{\rotatebox{270}{$\blacktriangle$}}~Демонстрация}};
	\end{tikzpicture}
	
	\begin{tikzpicture}[remember picture,overlay]
		\draw[rounded corners=0.2em, line width=0.5pt, darkgray]
		([xshift=10pt,yshift=-10pt]current page.north west) 
		rectangle 
		([xshift=-10pt,yshift=20pt]current page.south east);
	\end{tikzpicture}
	\vspace*{-1.5em}
}
\newcommand{\floor}[1]{\lfloor #1 \rfloor}